\documentclass[]{article}
\usepackage{amsmath}
\usepackage{graphicx}
\graphicspath{.}

%opening
\title{AERO 7970 - Multivariable Control of Uncertain Systems \\ Homework 2}
\author{Matt Boler}

\begin{document}

\maketitle

\section{Problem 1}

Given the system:

\begin{align*}
	\dot{x} &= \begin{bmatrix}
	0 & 1 & 0 & 0 & 0 \\
	0 & 0 & 1 & 0 & 0 \\
	0 & 0 & 0 & 1 & 0 \\
	0 & 0 & 0 & 0 & 1 \\
	0 & 0 & 0 & 0 & 0
	\end{bmatrix}
	x + \begin{bmatrix}
	0 \\
	0 \\
	0 \\
	0 \\
	1
	\end{bmatrix}u
	\\
	y &= Cx
\end{align*}

\begin{itemize}
	\item Check the controllability of the system
	\item Check the stability of the system
	\item What is the response to initial conditions?
\end{itemize}

\subsection{Solution}

\subsubsection{Controllability}

We form the controllability matrix $\mathbf{C}$ using Matlab's $ctrb(A, B)$ command.
The $\mathbf{A}$ matrix has $dim(\mathbf{A})=5$, and the $\mathbf{C}$ matrix has $rank(\mathbf{C}=5$.
Therefore the system is controllable.

\subsubsection{Stability}
From the $\mathbf{A}$ matrix being in Jordan form, we can see that $\lambda(\mathbf{A}) = 0 \forall \lambda$.
Therefore the system is marginally stable.

\subsubsection{Response to Initial Conditions}
The homogeneous response of the system is denoted by:
\begin{align*}
	\Phi(t) &= expm(\mathbf{A}t)x_0 = \begin{bmatrix}
	1 & t & \frac{t^2}{2} & \frac{t^3}{6} & \frac{t^4}{24} \\
	0 & 1 & t & \frac{t^2}{2} & \frac{t^3}{6} \\
	0 & 0 & 1 & t & \frac{t^2}{2} \\
	0 & 0 & 0 & 1 & t \\
	0 & 0 & 0 & 0 & 1 \\
	\end{bmatrix}x_0
\end{align*}

As we can see, the system is unstable to nonzero initial conditions.
\section{Problem 2}

Given the system:

\begin{align*}
\dot{x} &= \begin{bmatrix}
0 & 1 & 0 & 0 & 0 \\
0 & 0 & 1 & 0 & 0 \\
0 & 0 & 0 & 1 & 0 \\
0 & 0 & 0 & 0 & 1 \\
0 & 0 & 0 & 0 & 0
\end{bmatrix}
x + \mathbf{B}u
\\
y &= \begin{bmatrix}
1 & 0 & 0 & 0 & 0
\end{bmatrix}x
\end{align*}

\begin{itemize}
	\item Check the observability of the system
	\item Check the stability of the system
	\item What is the response to initial conditions?
\end{itemize}

\subsection{Solution}

\subsubsection{Observability}
We form the observability matrix $\mathbf{O}$ using Matlab's $obsv(A, C)$ command.
The $\mathbf{A}$ matrix has $dim(\mathbf{A})=5$, and the $\mathbf{O}$ matrix has $rank(\mathbf{O}=5$.
Therefore the system is observable.

\subsubsection{Stability}
As $\mathbf{A}$ is the same as in Problem 1, the same answer applies.

\subsubsection{Response to Initial Conditions}
As $\mathbf{A}$ is the same as in Problem 1, the same answer applies.

\section{Problem 3}

Given the system:

\begin{align*}
	\dot{x} = \begin{bmatrix}
	\dot{r} \\
	\ddot{r} \\
	\dot{\theta} \\
	\ddot{\theta} \\
	\dot{\phi} \\
	\ddot{\phi}
	\end{bmatrix} = 
	\begin{bmatrix}
	0 & 1 & 0 & 0 & 0 & 0 \\
	3\omega_0^2 & 0 & 0 & 2\omega_0r_0 & 0 & 0 \\
	0 & 0 & 0 & 1 & 0 & 0 \\
	0 & \frac{-2\omega_0 }{r_0} & 0 & 0 & 0 & 0	\\
	0 & 0 & 0 & 0 & 0 & 1 \\
	0 & 0 & 0 & 0 & -\omega_0^2 & 0
	\end{bmatrix}
	\begin{bmatrix}
	r \\
	\dot{r} \\
	\theta \\
	\dot{\theta} \\
	\phi \\
	\dot{\phi}
	\end{bmatrix} + 
	\begin{bmatrix}
	0 & 0 & 0 \\
	\frac{1}{m} & 0 & 0 \\
	0 & 0 & 0 \\
	0 & \frac{1}{mr_0} & 0 \\
	0 & 0 & 0 \\
	0 & 0 & \frac{1}{mr_0}
	\end{bmatrix}
	\begin{bmatrix}
	u_r \\
	u_{\theta} \\
	u_{\phi}
	\end{bmatrix}
\end{align*}

\begin{itemize}
	\item Determine the stability, controllability, and observability properties
	\item Find the following transfer functions:
	\begin{align*}
		\frac{r(s)}{u_r(s)}, \frac{\phi(s)}{u_{\phi}(s)}, \frac{\theta(s)}{u_{\theta}(s)}
	\end{align*}
\end{itemize}

\subsection{Solution}

\subsubsection{Stability, Controllability, Observability}

The system has six eigenvalues, all on the imaginary axis.
Thus the system is marginally stable.

The system is 6-dimensional, and both the controllability and observability matrices are rank 6.
Thus the system is both controllable and observable.

\subsubsection{Transfer Functions}

We split the $\mathbf{A}$ matrix into the differential equations it represents:
\begin{align*}
	\dot{r} &= \dot{r} \\
	\ddot{r} &= 2\omega_0^2r + 2\omega_0r_0 \dot{\theta} + \frac{1}{m}u_r \\
	\dot{\theta} &= \dot{\theta} \\
	\ddot{\theta} &= \frac{-2\omega_0}{r_0} \dot{r} + \frac{1}{mr_0} u_{\theta} \\
	\dot{\phi} &= \dot{\phi} \\
	\ddot{\phi} &= -\omega_0^2\phi + \frac{1}{mr_0}u_{\phi}
\end{align*}

We solve for $\frac{\phi(s)}{u_{\phi}(s)}$ by taking the inverse laplace of the $\ddot{\phi}$ equation.
Rearranging gives:

\begin{align*}
	\frac{\phi(s)}{u_{\phi}(s)} = \frac{\frac{1}{mr_0}}{s^2 + \omega_0^2} 
\end{align*}

As the differential equation for $\ddot{r}$ includes a $\dot{\theta}$ term, we solve for $\theta(s)$ in terms of $r(s)$ and remove the input $u_{\theta}(s)$.
Rearranging gives:
\begin{align*}
	\frac{r(s)}{u_r(s)} &= \frac{\frac{1}{m}}{s^2 + \omega_0^2}
\end{align*}

Similarly, the differential equation for $\ddot{\theta}$ includes an $\dot{r}$ term.
Following the same process and eliminating $u_r(s)$ gives:

\begin{align*}
	\frac{\theta(s)}{u_{\theta}(s)} &= \frac{s^2 - 3\omega_0^2}{mr_0s^2 (s^2 + \omega_0^2)}
\end{align*}

\section{Problem 4}

Given the system:

\begin{align*}
	G(s) = \frac{s+2}{s^2 + 7s + 12}
\end{align*}

Show that $z=-2$ is a transmission zero using the generalized eigenvalue problem.

\subsection{Solution}

To show that $z=-2$ is a transmission zero, it must solve

\begin{align*}
	\begin{bmatrix}
	z\mathbf{I} - \mathbf{A} & -\mathbf{B} \\
	\mathbf{C} & \mathbf{D}
	\end{bmatrix}
	\begin{bmatrix}
	\mathbf{x}_0 \\
	u_0
	\end{bmatrix} = 0
\end{align*}

for the state space system:

\begin{align*}
	\mathbf{A} &= \begin{bmatrix}
	-7 & -12 \\
	1 & 0
	\end{bmatrix} \\
	\mathbf{B} &= \begin{bmatrix}
	1 \\
	0
	\end{bmatrix}\\
	\mathbf{C} &= \begin{bmatrix}
	1 & 2
	\end{bmatrix} \\
	\mathbf{D} &= 0
\end{align*}

The constructed matrix system is then:

\begin{align*}
	\begin{bmatrix}
	5 & 12 & -1 \\
	-1 & -2 & 0 \\
	1 & 2 & 0
	\end{bmatrix}
	\begin{bmatrix}
	\mathbf{x}_0 \\
	u_0
	\end{bmatrix} = 0
\end{align*}

Clearly, rows 2 and 3 of this matrix are not linearly dependent, and as such the system has lost rank.
Thus $z=-2$ is a transmission zero.
Solving the matrix system for $\mathbf{x}_0$ and $u_0$ gives $x_1 = -2x_2$ and $u_0 = 2x_2$.


\end{document}
