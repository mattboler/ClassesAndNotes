\documentclass[]{article}
\usepackage{amsmath}
\usepackage{graphicx}
\graphicspath{.}

%opening
\title{MECH6970 Fundamentals of Nav. and Guidance \\ Homework 1}
\author{Matt Boler}

\begin{document}

\maketitle



\section{Problem 1}

\subsection{Part A}

The satellite's geodetic coordinates are 

\begin{align*}
	\overrightarrow{P}^g &= \begin{bmatrix}
	19.3602 \\
	-99.0719 \\
	2.1986 \times 10^7
	\end{bmatrix}
\end{align*}

\subsection{Part B}

The satellite is located over Mexico City.

\subsection{Part C}

The satellite's speed in $km/s$ is

\begin{align*}
	s &= || \overrightarrow{V}^e ||_2 \times 10^{-3} \\
	&= 3.9999 \frac{km}{s}
\end{align*}

\subsection{Part D}

The transformation matrix is

\begin{align*}
	^{ecef}\overrightarrow{\mathbf{C}}^{ned} &= \begin{bmatrix}
	cos(\alpha) & 0 & sin(\alpha) \\
	0 & 1 & 0 \\
	-sin(\alpha) & 0 & cos(\alpha)
	\end{bmatrix} \begin{bmatrix}
	cos(\lambda) & sin(\lambda) & 0 \\
	-sin(\lambda) & cos(\lambda) & 0 \\
	0 & 0 & 1
	\end{bmatrix}
\end{align*}

\begin{align*}
	\overrightarrow{V}^{ned} &= ^{ecef}\overrightarrow{\mathbf{C}}^{ned} \overrightarrow{V}^{ecef} \\
	&= \begin{bmatrix}
	2.8284 \\
	2.8284 \\
	0.0038
	\end{bmatrix} \times 10^3 \frac{m}{s}
\end{align*}

\subsection{Part E}

To find the az/el, we convert to ECEF and find the vector between the observer and satellite, and then convert that vector to a local NED frame to get our az/el

\begin{align*}
	\overrightarrow{P}^{ecef}_{obsv} &= \begin{bmatrix}
	1.2733 \\ -6.2526 \\-0.0156
	\end{bmatrix} \times 10^6 \\
	\overrightarrow{P}^{ecef}_{rel} &= \overrightarrow{P}^{ecef}_{obsv} - \overrightarrow{P}^{ecef}_{sat} \\
	\overrightarrow{P}^{ned}_{rel} &= ^{ecef}\overrightarrow{\mathbf{C}}^{ned}_{obsv} \overrightarrow{P}^{ecef}_{rel} \\
	&= \begin{bmatrix}
	0.9491 \\
	-0.9408 \\
	-1.8650
	\end{bmatrix} \times 10^7
\end{align*}

Our az/el is then

\begin{align*}
	\psi &= tan^{-1}(\frac{\Delta_y}{\Delta_x}) \\
	&= -44.7494deg \\
	\gamma &= tan^{-1}( \frac{-\Delta_z}{sqrt( \Delta_x^2 + \Delta_y^2 )} ) \\
	&= 54.3760deg
\end{align*}

\section{Problem 2}

In addition to the assumptions made by the problem, I assume a nonrotating earth.
We consider two paths: parallel to the equator and crossing over the north pole.

\subsection{Parallel to the Equator}

The distance travelled is an arc across a sphere of radius $R_{earth} + alt$.
We find the arc length via

\begin{align*}
	L_1 &= \frac{\Delta_{lon}}{360} * circumference \\
	&= \frac{180}{360} * cos(lat) * (R_{earth} + alt) * 2 * \pi \\
	&= 1.4191 \times 10^7
\end{align*}

We divide this by speed to get the time of flight

\begin{align*}
	t_1 &= \frac{L_1}{s \times 10^3} \\
	&= 17.7386 hr
\end{align*}

\subsection{Over the North Pole}

\begin{align*}
	L_2 &= \frac{\Delta_{lat}}{360} * circumference \\
	&= \frac{90}{360} * R_{earth} * 2 * \pi \\
	&= 1.0034 \times 10^7
\end{align*}

\begin{align*}
	t_2 &= \frac{L_2}{s \times 10^3} \\
	&= 12.5431hr
\end{align*}

So flying directly over the north pole is the fastest route.

\section{Problem 3}

If the drone is always flying at 5 degrees off of North, then it will circle the North pole for eternity.
Otherwise, if it just starts at an error of 5 degrees then it will complete one full circumnavigation of the earth before returning to its origin.
We show this by

\begin{align*}
	t &= \frac{L}{v} \\
	&= \frac{2 * \pi * R_{earth}}{v_{drone}} \\
	&= 400.7502 hrs
\end{align*}

\subsection{Textbook Problem 1}

\subsection{Degrees}

If 1 minute of latitude is equivalent to 1852 meters, then the table below shows accuracies for varying numbers of significant digits

\begin{center}
	\begin{tabular}{|c|c|} \hline
		Decimal Places & Accuracy in Meters \\ \hline
		0 & 1852 \\
		1 & 185.2 \\
		2 & 18.52 \\
		3 & 1.852 \\
		4 & 0.1852 \\
		5 & 0.01852 \\
		6 & 0.001852 \\ \hline
	\end{tabular}
\end{center}

6 decimal places are needed.

\subsection{Arc-Seconds}
% Latitude: deg min sec
% min = 1/60 min
% sec = 1/3600 min

\begin{center}
	\begin{tabular}{|c|c|} \hline
		Decimal Places & Accuracy in Meters \\ \hline
		0 & 0.514 \\
		1 & 0.0514 \\
		2 & 0.00514 \\ \hline
	\end{tabular}
\end{center}

Two decimal places are needed in the arc-seconds field for a sub 1cm solution.

\section{Textbook Problem 2}

Assuming a spherical earth with a radius of 6371km, the circumference of said earth is 

\begin{align*}
	c_e &= 2 * \pi * 6371 \\
	&= 40,030km
\end{align*}

Moving at 885 km/h for 8h results in a total distance of 

\begin{align*}
	d &= s \times t \\
	&= 885 * 8 \\
	&= 7,080km
\end{align*}

which is $0.1769\%$ of the circumference of the earth, and thus approximately 63.67 degrees of motion on a spherical earth.
Since we travel at a heading of $45$ degrees off North, this motion is split into 

\begin{align*}
	d_N &= 63.67 \times sin(45) \\
	&= 45.0215 deg \\
	d_E &= 63.67 \times cos(45) \\
	&= 45.0215 deg
\end{align*}

Which places us at approximately 90 degrees North, 75 degrees West at an altitude of 10km.
In other words, we are at the North Pole.

\section{Textbook Problem 3}

\subsection{Part A}

We use the formula 

\begin{align*}
	\dot{\rho} &= c \times (1 - \frac{f_{trans} + f_{shift}}{f_{trans}})
\end{align*}

to find our range rates

\begin{align*}
	\dot{\rho}_0 &= 99.5037 m/s \\
	\dot{\rho}_1 &= 99.5133 m/s \\
	\dot{\rho}_2 &= 99.5229 m/s
\end{align*}

\subsection{Part B}

Our linear equations are

\begin{align*}
	x_1 &= x_0 + v_x \times \Delta_t \\
	x_2 &= x_0 + v_x \times 2*\Delta_t
\end{align*}

\subsection{Part C}

Our nonlinear equations are

\begin{align*}
	\theta &= tan^{-1}(\frac{y_0}{x_0 + \dot{x}\Delta_t}) \\
	\dot{x} &= \dot{\rho}cos(\theta)
\end{align*}


\section{Problem 4}

\subsection{Part A}

Given $d_1 = 550m = \rho_1 - b$, $d_2 = 500m = \rho_2 - b$, we find the true ranges and bias as such

\begin{align*}
	1000 &= \rho_1 + \rho_2 \\
	&=  d_1 + b + d_2 + b \\
	&= 550 + b + 500 + b \\
	&= 1050 + 2b \\
\end{align*}

so $b = -25m$. 
Thus the true position of the observer is $525m$ from PL1.

\subsection{Part B}

Given $d_1 = 400m = \rho_1 - b$, $d_2 = 1400m = \rho_2 - b$, we find the true ranges and bias as such

\begin{align*}
1000 &= \rho_1 + \rho_2 \\
&=  d_1 + b + d_2 + b \\
&= 400 + b + 1400 + b \\
&= 1400 + 2b \\
\end{align*}

so $b = -400m$. 
Thus the true position of the observer is $0m$ from PL1.

\section{Textbook Problem 5}

\subsection{Part A}

Given that our only measurement is our odometer measurement, our estimate of position is our odometer reading of 56km.

\subsection{Part B}

Since the bus leaves on the hour and moves at 3km/min, it should have traveled

\begin{align*}
	d &= v \times t \\
	&= 21 * 3 \\
	&= 63km
\end{align*}

so according to our timing we are 63km from town A.

\subsection{Part C}

Our two equations for position are

\begin{align*}
	m_{odom} &= d + e_{odom} \\
	m_{red bus} &= s_{red bus} \times (t + b_{clock})
\end{align*}

\subsection{Part D}

Since the bus leaves on the hour and moves at 2.5km/min, our new navigation equations are

\begin{align*}
	m_{odom} &= d + e_{odom} \\
	m_{red bus} &= s_{red bus} \times (t_{red bus} + b_{clock}) \\
	120 - m_{blue bus} &=  s_{blue bus} \times (t_{blue bus} + b_{clock})
\end{align*}

We can solve for our clock bias with

\begin{align*}
	120 &= d_A + d_B \\
	&= m_{red bus} - s_{red bus}b + m_{blue bus} - s_{blue bus}b \\
	&= 63km - 3b + 62.5 - 2.5b \\
	120 &= 125.5 - 5.5b
\end{align*}

so our clock bias is 1min.

Our position is then 

\begin{align*}
	d &= s_{red bus} \times t_{red bus} \\
	&= m_{red bus} - s_{red bus}b \\
	&= 60km
\end{align*}

\subsection{Part E}

Your solution would be the same, as the added clock biases would cancel out.

\subsection{Part F}

\subsection{L1}

You can't estimate either with just the L1 cabs, because you have no way of differentiating the cabs and thus can't tell when the cab left town A.

\subsection{L1 and L2}

With the L1 and L2 cabs, you still can't calculate either because you have no way to know how long each cab has been driving.
You can figure out how far you are between whole km intervals from the offset in L1 and L2 cabs arriving, but once you reach a new km the offsets wrap around.

\end{document}
