\documentclass[]{article}
\usepackage{amsmath}
\usepackage{graphicx}
\graphicspath{.}
%opening
\title{ELEC7500 Project 5}
\author{Matthew Boler}

\begin{document}

\maketitle

\section{Modeling}

Our task is to analyze and design a control system for the inverted pendulum.
The variables and parameters are given in Table \ref{table:variables}.

\begin{table}[h]
\begin{center}
	\begin{tabular}{| l | l | l |}
		\hline
		Symbol & Description & Value \\
		\hline
		$m$ & Your body mass & $80$kg \\
		$l$ & Length of pendulum & $0.5$m \\
		$M$ & Mass of cart & $45$kg \\
		$H$ & Cart rolling friction coefficient & $4.5\frac{Ns}{m}$ \\
		$g$ & Gravitational constant & $9.8\frac{m}{s^2}$ \\
		$\theta$ & Pendulum angle & varies \\
		$x$ & Cart position & varies \\
		$F$ & Force exerted on cart & varies\\
		\hline
	\end{tabular} \\

\end{center}
\caption{System variables and parameters}
\label{table:variables}
\end{table}

The dynamics of the system are given by Equations \ref{eq:stateeq1} and \ref{eq:stateeq2}.

\begin{equation}
	\ddot{x} \cos(\theta) + l\ddot{\theta} - g\sin(\theta) = 0 \label{eq:stateeq1}
\end{equation}

\begin{equation}
	(M + m)\ddot{x} + ml\ddot{\theta}\cos(\theta) - ml\dot{\theta}^2\sin(\theta) = F - H\dot{x} \label{eq:stateeq2}
\end{equation}

We define the system state to be 
\begin{align*}
\textbf{z} = \begin{bmatrix}
\theta \\
\dot{\theta} \\
x \\
\dot{x}
\end{bmatrix}
\end{align*}
with the system input $u = F$. We rewrite the dynamic equations in state-variable form

\begin{align*}
	\dot{\mathbf{z}} &= f(\mathbf{z}, u)
\end{align*}

\section{Equilibrium State}

\section{Linearization}

\section{State Stability}

\section{Transfer Function}

\section{Other Properties}

\section{State Feedback}

\section{State Estimation}

\section{Computer Simulations}

\subsection{Linear Model with State Feedback}

\subsection{Linear Model with Estimated State Feedback}

\subsection{Nonlinear Model with Linear State Feedback}

\subsection{Nonlinear Model with Estimated State Feedback}

\section{Revisiting the Separation Principle}



\end{document}
